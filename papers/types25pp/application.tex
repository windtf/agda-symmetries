% !TeX root = types25pp-sort.tex

\section{Sorting Functions}
\label{sec:application}
\label{sec:sorting}

We will now put to work the universal properties of our types of (ordered) lists and unordered lists,
to define operations on them systematically, which are mathematically sound, and reason about them.
%
First, we explore definitions of various operations on both free monoids and free commutative monoids.
%
By univalence (and the structure identity principle), everything henceforth holds for any presentation of free monoids
and free commutative monoids.
%
We use $\FF(A)$ to denote the free monoid or free commutative monoid on $A$,
$\LL(A)$ to exclusively denote the free monoid construction,
and $\MM(A)$ to exclusively denote the free commutative monoid construction.

%
For example $\term{length}$ is a common operation defined inductively for $\List$,
but usually, properties about $\term{length}$, such as,
$\term{length}(\xs \doubleplus \ys) = \term{length}(\xs) + \term{length}(\ys)$,
are proven separately after defining it.
%
In our framework of free algebras, where the $\ext{(\blank)}$ operation is a correct-by-construction homomorphism,
we can define operations like $\term{length}$ directly by universal extension,
which also gives us a proof that they are homomorphisms for free.
%
A further application of the universal property is to prove that two different types are equal,
by showing they both satisfy the same universal property as in~\cref{lem:free-algebras-unique},
which is desirable especially when proving a direct equivalence between the two types turns out
to be a difficult exercise in combinatorics.

% !TEX root = types25pp-sort.tex

\subsection{Prelude}
\label{sec:prelude}

Any presentation of free monoids or free commutative monoids has a $\term{length} : \FF(A) \to \Nat$ function,
where $\Nat$ carries the additive monoid structure $(0,+)$,
which is also a commutative monoid structure since addition is commutative.
%
\begin{definition}[\alink{definition}{length}{length}]
      \label{def:length}
      The length homomorphism is defined as
      \(
      \ext{(\lambda x.\, 1)} : \FF(A) \to \Nat
      \).
\end{definition}
%
Further, any presentation of free monoids or free commutative monoids has an
element membership predicate ${\blank\in\blank} : A \to \FF(A) \to \hProp$,
for any set $A$.
%
Here, we use the fact that $\hProp$ forms a (commutative) monoid under
disjunction and falsehood $(\bot, \vee)$.
%
\begin{definition}[\alink{definition}{Membership~$\in$}{membership}]
      \label{def:membership}
      The membership predicate on a set $A$ for any element $x:A$ is
      \(
      {x\in\blank} \defeq \ext{\yo_{A}(x)} : {\FF(A) \to \hProp}
      \),
      where we define
      \(
      \yo_A(x) \defeq {\lambda y.\, {x \id y}} : {A \to \hProp}
      \).
\end{definition}
%
$\yo$ is formally the Yoneda map under the ``types are groupoids'' correspondence,
where $x:A$ is being sent to its representable in the Hom-groupoid (formed by the identity type), of type $\hProp$.
%
Note that the proofs of (commutative) monoid laws for $\hProp$ use equality,
which requires the use of univalence (or at least, propositional extensionality).
%
By construction, this membership predicate satisfies its homomorphic properties,
which are colloquially the properties of inductively defined de Bruijn indices.

We note that $\hProp$ is actually one type level higher than $A$.
To make the type level explicit, $A$ is of type level $\ell$, and since $\hProp_\ell$
is the type of all types $X : \Set_\ell$ that are mere propositions, $\hProp_\ell$ has
type level $\ell + 1$. We do not assume any propositional resizing axioms~\cite{voevodskyResizingRulesTheir2011},
and use level polymorphism $\ext{(\blank)}$ in our formalisation to accommodate this.

Any presentation of free (commutative) monoids $\FF(A)$ also supports the
$\term{Any}$ and $\term{All}$ predicates, which allow lifting a predicate $A \to \hProp$ (on $A$),
to \emph{any} or \emph{all} elements of $\xs : \FF(A)$, respectively.
%
We note that $\hProp$ forms a (commutative) monoid in two different ways:
$(\bot,\vee)$ and $(\top,\wedge)$ (disjunction and conjunction),
which are the two different ways of getting $\term{Any}$ and $\term{All}$,
respectively, by extension.
\begin{definition}[$\term{Any}$ and $\term{All}$]
      \label{def:any-all}
      \begin{gather*}
            \type{Any}(P) \defeq \ext{P} : \FF(A) \to (\hProp, \bot, \vee)
            \qquad
            \type{All}(P) \defeq \ext{P} : \FF(A) \to (\hProp, \top, \wedge)
      \end{gather*}
\end{definition}
%
Note that Cubical Agda has problems with indexing over HITs~\cite[\S~8]{ProperSupportInductive,alexandruIntrinsicallyCorrectSorting2023}
hence it is preferable to program with our universal properties, such as when defining $\term{Any}$ and $\term{All}$,
because the indexed-inductive definitions of these predicates get stuck on $\term{transp}$ terms.

There is a $\term{head}$ function on lists, which is a function that returns the first element of a non-empty list.
%
Formally, this is a monoid homomorphism from $\LL(A)$ to $1 + A$.
%
\begin{definition}[\alink{definition}{$\term{head}$}{head}]
      \label{def:head-free-monoid}
      The head homomorphism is defined as
      \(
      \term{head} \defeq \ext{\inr} : \LL(A) \to 1 + A
      \),
      where the monoid structure on $1 + A$ has unit
      \(
      e \defeq \inl(\ttt) : 1 + A
      \),
      and multiplication picks the leftmost element that is defined.
      \[
            \begin{array}{rclcl}
                  \inl(\ttt) & \oplus & b & \defeq & b       \\
                  \inr(a)    & \oplus & b & \defeq & \inr(a) \\
            \end{array}
      \]
\end{definition}
%
This monoid operation $\oplus$ is not commutative,
since swapping the input arguments to $\oplus$ would return the leftmost or rightmost element.
%
To make it commutative would require a canonical way to pick between a choice of two elements --
this leads us to the next section.
 %% 2 pages
% !TEX root = types25pp-sort.tex

\subsection{Total orders}
\label{sec:total-orders}

First, we recall the axioms of a total order or linear order $\leq$ on a set $A$.
\begin{definition}[\alink{definition}{Total order}]
    \label{def:total-order}
    A total order on a set $A$ is a relation $\leq : A \to A \to \hProp$ that satisfies:
    \begin{itemize}
        \item reflexivity: $x \leq x$,
        \item transitivity: if $x \leq y$ and $y \leq z$, then $x \leq z$,
        \item antisymmetry: if $x \leq y$ and $y \leq x$, then $x = y$,
        \item totality: $\forall x, y$, either $x \leq y$ or $y \leq x$.
    \end{itemize}
    A \emph{decidable} total order requires the $\leq$ relation to be decidable:
    \begin{itemize}
        \item decidable totality: $\forall x, y$, we have $x \leq y + \neg(x \leq y)$.
    \end{itemize}
\end{definition}
Note that \emph{either-or} means a (truncated) logical disjunction.
In the context of this paper, we want to make a distinction between
``decidable total order'' and ``total order''.
The decidability axiom strengthens the totality axiom,
where we have either $x \leq y$ or $y \leq x$ merely as a proposition,
but decidability allows us to produce a witness if $x \leq y$ is true.
\begin{proposition}[\alink{proposition}{}]
    \label{prop:decidable-total-order}
    In a decidable total order, it holds that ${\forall x, y}, \ps{x \leq y} + \ps{y \leq x}$.
    Further, this makes $A$ discrete, that is ${\forall x, y}, \ps{x \id y} + \ps{x \neq y}$.
\end{proposition}
\begin{proof}
    We decide if $x \leq y$ and $y \leq x$, and by cases:
    \begin{itemize}
        \item
              if $x \leq y$ and $y \leq x$: by antisymmetry, $x = y$.
        \item
              if $\neg(x \leq y)$ and $y \leq x$: assuming $x = y$ leads to a contradiction, hence $x \neq y$.
        \item
              if $x \leq y$ and $\neg(y \leq x)$: similar to the previous case.
        \item
              if $\neg(x \leq y)$ and $\neg(y \leq x)$: by totality, either
              $x \leq y$ or $y \leq x$, which leads to a contradiction.
    \end{itemize}
\end{proof}
We also recall the axioms of a strict total order $<$ on $A$.
\begin{definition}[\alink{definition}{Strict total order}]
    \label{def:strict-total-order}
    A strict total order on a set $A$ is a relation $< : A \to A \to \hProp$ that satisfies:
    \begin{itemize}
        \item irreflexivity: $\neg(x < x)$,
        \item transitivity: if $x < y$ and $y < z$, then $x < z$,
        \item asymmetry: if $x < y$, then $\neg(y < x)$,
        \item cotransitivity: $\forall x, y, z$, if $x < z$, then either $x < y$ or $y < z$.
        \item connectedness: $\forall x, y$, if $\neg(x < y)$ and $\neg(y < x)$, then $x = y$.
    \end{itemize}
    A \emph{decidable} strict total order requires the $<$ relation to be decidable:
    \begin{itemize}
        \item decidability: $\forall x, y$, we have $x < y + \neg(x < y)$.
    \end{itemize}
\end{definition}

\begin{proposition}[\alink{proposition}{}]
    \label{prop:decidable-strict-total-order}
    In a decidable strict total order, it holds that ${\forall x, y}, \ps{x < y} + \ps{y < x}$.
    Further, this makes $A$ discrete, that is ${\forall x, y}, \ps{x \id y} + \ps{x \neq y}$.
\end{proposition}
\begin{proof}
    We decide if $x < y$ and $y < x$, and by cases:
    \begin{itemize}
        \item if $\neg(x < y)$ and $\neg(y < x)$: by connectedness, $x = y$.
        \item if $x < y$ or $y < x$: by irreflexivity, $x \neq y$.
    \end{itemize}
\end{proof}

\begin{proposition}[\alink{proposition}{}]
    \label{prop:decidable-strict-total-order}
    The set of decidable strict total orders on $A$ and the set of decidable total orders on $A$ are equivalent.
\end{proposition}
\begin{proofsketch}
    Given a decidable total order $\leq$ on $A$, by~\cref{prop:decidable-total-order},
    we map it to a decidable strict total order $<$ on $A$ by
    $x < y \defeq (x \leq y) \times (x \neq y)$. Vice versa, by~\cref{prop:decidable-strict-total-order}
    given a decidable strict total order $<$ on $A$,
    we map it to a decidable total order $\leq$ on $A$ by
    $x \leq y \defeq (x < y) + (x = y)$.
    These maps are inverses of each other.
\end{proofsketch}

%
\noindent
An equivalent way of defining a total order is using a binary meet operation, without starting from an ordering relation.
\begin{definition}[\alink{definition}{Meet semi-lattice}]
    \label{def:meet-semi-lattice}
    A meet semi-lattice is a set $A$ with a binary operation $\blank\meet\blank : A \to A \to A$ that is:
    \begin{itemize}
        \item idempotent: $x \meet x \id x$,
        \item associative: $(x \meet y) \meet z \id x \meet (y \meet z)$,
        \item commutative: $x \meet y \id y \meet x$.
    \end{itemize}
    A \emph{strongly-connected} meet semi-lattice further satisfies:
    \begin{itemize}
        \item strong-connectedness: $\forall x, y$, either $x \meet y \id x$ or $x \meet y \id y$.
    \end{itemize}
    A \emph{decidable} strongly-connected meet semi-lattice strengthens this to:
    \begin{itemize}
        \item decidable strong-connectedness: ${\forall x, y}, \ps{x \meet y \id x} + \ps{x \meet y \id y}$.
    \end{itemize}
\end{definition}

\begin{proposition}[\alink{proposition}{}]
    \label{prop:total-order-meet-semi-lattice}
    A total order $\leq$ on a set $A$ is equivalent to a strongly-connected meet semi-lattice structure on $A$.
    Further, a decidable total order on $A$ induces a decidable strongly-connected meet semi-lattice structure on $A$.
    \vc{equivalence?}
\end{proposition}
\begin{proofsketch}
    Given a (mere) total order $\leq$ on a set $A$,
    we define ${x \meet y} \defeq \term{if} x \leq y \term{then} x \term{else} y$.
    %
    Crucially, this map is \emph{locally-constant}, allowing us to eliminate from an $\hProp$ to an $\hSet$.
    %
    Meets satisfy the universal property of products, that is,
    ${c \leq a \meet b} \Leftrightarrow {c \leq a} \land {c \leq b}$,
    and the axioms follow by calculation using $\yo$-arguments.
    %
    Conversely, given a meet semi-lattice, we define $x \leq y \defeq x \meet y \id x$,
    which defines an $\hProp$-valued total ordering relation.
    %
    If the total order is decidable, we use the discreteness of $A$ from~\cref{prop:decidable-total-order}.
\end{proofsketch}
%
Finally, tying this back to~\cref{def:head-free-monoid}, we have the following for free commutative monoids.
\begin{definition}[\alink{definition}{$\term{head}$}]
    \label{def:head-free-commutative-monoid}
    Assume a total order $\leq$ on a set $A$.
    We define a commutative monoid structure on $1 + A$,
    with unit \(e \defeq \inl(\ttt) : 1 + A\), and multiplication defined as:
    \[
        \begin{array}{rclcl}
            \inl(\ttt) & \oplus & b          & \defeq & b                         \\
            \inr(a)    & \oplus & \inl(\ttt) & \defeq & \inr(a)                   \\
            \inr(a)    & \oplus & \inr(b)    & \defeq & \inr(a \meet b) \enspace.
        \end{array}
    \]
    This gives a homomorphism \({\term{head} \defeq \ext{\inr}} : {\MM(A) \to 1 + A}\),
    which picks out the \emph{least} element of the free commutative monoid.
\end{definition}

\subsection{Sorting functions}
\label{sec:sorting}

The free commutative monoid is also a monoid, hence, there is a canonical monoid homomorphism
$q : \LL(A) \to \MM(A)$, which is given by $\ext{\eta_A}$, the extension of the unit $\eta_A : A \to \MM(A)$.
%
Since $\MM(A)$ is (upto equivalence), a quotient of $\LL(A)$ by symmetries (or a permutation relation),
it is a surjection (in particular, a regular epimorphism in $\Set$ as constructed in type theory).
%
Concretely, $\quotient$ simply includes the elements of $\LL(A)$ into equivalence classes of lists in $\MM(A)$,
which ``forgets'' the order that was imposed by the indexing of the list.

Classically, assuming the Axiom of Choice would allow us to construct a section (right-inverse, in $\Set$) to the surjection $\quotient$,
that is,
a function $s : \MM(A) \to \LL(A)$ such that $\forall x.\, q(s(x)) \id x$.
%
Or in informal terms, given the surjective inclusion into the quotient,
a section (uniformly) picks out a canonical representative for each equivalence class.
%
The core question we want to study is the existence of $\ssection$ in a constructive
setting, or equivalently, whether the order factored out by the symmetry quotient
can be constructively recovered.
\begin{figure}
    \centering
    \scalebox{1.0}{
        % https://q.uiver.app/#q=WzAsMixbMCwwLCJcXExMKEEpIl0sWzMsMCwiXFxNTShBKSJdLFsxLDAsInMiLDAseyJjdXJ2ZSI6LTF9XSxbMCwxLCJxIiwwLHsiY3VydmUiOi0xfV1d
        \begin{tikzcd}[ampersand replacement=\&,cramped]
            {\LL(A)} \&\&\& {\MM(A)}
            \arrow["\ssection", curve={height=-10pt}, from=1-4, to=1-1]
            \arrow["\quotient", two heads, from=1-1, to=1-4]
        \end{tikzcd}
    }
    \caption{Relationship of $\LL(A)$ and $\MM(A)$}
    \label{fig:enter-label}
\end{figure}
%
Viewing the quotienting relation as a permutation relation (from~\cref{sec:cmon:qfreemon}), a section $\ssection$ to $\quotient$ has to pick out
canonical representatives of equivalence classes generated by permutations.
%
Using $\SList$ as an example,
$s(x \cons y \cons \xs) \id s(y \cons x \cons \xs)$ for any $x, y : A$ and $\xs : \SList(A)$,
by $\term{swap}$.
%
Since $\forall \xs.\,q(s(\xs)) \id \xs$, $\ssection$ must preserve all the elements of $\xs$.
It cannot be a trivial function such as $\lambda\,\xs. []$ -- it must produce a permutation of the elements of $\ssection$.
%
But to place these elements side-by-side in the list, $\ssection$ must somehow impose an order on $A$
(invariant under permutation), turning unordered lists of $A$ into ordered lists of $A$.
%
Axiom of Choice (AC) giving us a section $\ssection$ to $\quotient$ ``for free'' is analagous to how
AC implies the well-ordering principle, which states every set can be well-ordered.
%
Thus, if AC was assumed, we could easily recover an order on $A$ from the section $\ssection$.
%
Instead we study how to constructively define such a section, and in fact,
that is exactly the extensional view of a sorting algorithm,
and the implications of its existence is that $A$ can be ordered, or sorted.

\subsubsection{Section from Order}
\label{sec:sort-section-from-order}

\begin{proposition}[\alink{proposition}{}]
    Assume a decidable total order on $A$. There is a sort function $\ssection: \MM(A) \to \LL(A)$
    which constructs a section to $\quotient : \LL(A) \twoheadrightarrow \MM(A)$
\end{proposition}

\begin{proofsketch}
    We may construct such a sort function by implementing any sorting algorithm.
    In our formaliziation we chose insertion sort,
    because it can be defined easily using the inductive structure of $\SList(A)$ and $\List(A)$.
    To implement other sorting algorithms like mergesort,
    other representations such as $\Bag$ and $\Array$ would be preferable,
    as explained in~\cref{sec:cmon:bag}.
    To see how this proposition holds: $q(s(\xs))$ first orders an unordered list $\xs$ by $\ssection$,
    then discards the order again by $\quotient$ --
    imposing and then forgetting an order on $\xs$ simply \emph{permutes} its elements,
    which proves $\quotient \comp \ssection = \idfunc$.
\end{proofsketch}

\subsubsection{Order from Section}
\label{sec:sort-order-from-section}

The previous section allowed us to construct a section, but an arbitrary section may not be a sorting function.
%
To show a section is indeed a sort function, we need to show the section imposes some total order on $A$ which it sorts by.
%
Indeed, just by the virtue of $\ssection$ being a section,
we can \emph{almost} construct a total-order on the carrier set.

\begin{definition}[\alink{definition}{$\term{least}$}]
    \label{def:least}
    Given a section $\ssection$, we define:
    \[
        x \leqs y \defeq \term{head}(s(\bag{x, y})) = \inr(x) \enspace.
    \]
\end{definition}
%
That is, we take the two-element bag $\bag{x, y}$,
``sort'' it by $\ssection$, and test if the $\term{head}$ element is $x$.
%
Note, this is equivalent to $x \leqs y \defeq s\bag{x, y} = [x,y]$,
because $\ssection$ preserves length, and the second element is forced to be $y$.
%
% \begin{proposition}
%     $\leqs$ is decidable iff $A$ has decidable equality.
% \end{proposition}

\begin{proposition}[\alink{proposition}{}]
    \label{sort:almost-total}
    $\leqs$ is reflexive, antisymmetric, and total.
\end{proposition}
\begin{proof}
    For all $x$, $\term{least}(\bag{x, x})$ must be $\inr(x)$, therefore $x \leqs x$, giving reflexivity.
    For all $x$ and $y$, given $x \leqs y$ and $y \leqs x$,
    we have $\term{least}(\bag{x, y}) = \inr(x)$ and $\term{least}(\bag{y, x}) = \inr(y)$.
    Since $\bag{x, y} = \bag{y, x}$, $\term{least}(\bag{x, y}) = \term{least}(\bag{y, x})$,
    therefore we have $x = y$, giving antisymmetry.
    For all $x$ and $y$, $\term{least}(\bag{x, y})$ is merely either $\inr(x)$ or $\inr(y)$,
    therefore we have merely either $x \leqs y$ or $y \leqs x$, giving totality.
\end{proof}
%
A crucial observation is that $\ssection$ correctly orders 2-element bags,
but it does not necessarily sort bags with 3 or more elements.
\begin{proposition}
    \label{prop:counterexample-transitivity}
    $\leqs$ is not necessarily transitive.
\end{proposition}
\begin{proof}
    We give a counterexample of an $\ssection$ that violates transitivity.
    Consider the section $s : \SList(\Nat) \to \List(\Nat)$ defined as follows.
    First, we define a sort function
    $\term{sort} : \SList(\Nat) \to \List(\Nat)$ which sorts $\SList(\Nat)$ ascendingly.
    Then we use $\term{sort}$ to construct $\ssection$.
    \begin{align*}
        \ssection(\xs) & = \begin{cases}
                       \term{sort}(\xs)                 & \text{if $\term{length}(\xs)$ is odd} \\
                       \term{reverse}(\term{sort}(\xs)) & \text{otherwise}
                   \end{cases}
    \end{align*}
    %
    Now we have $\ssection([2,3,1]) = [1,2,3]$ but $\ssection([2,3,1,4]) = [4,3,2,1]$.
\end{proof}
%
We will enforce additional constraints on the \emph{image} of $\ssection$, to turn it into a correct sort function.
\begin{definition}[\alink{definition}{$\blank\in\im{\ssection}$}]
    \label{def:in-image}
    The fiber of $\ssection$ at a point in the codomain~$\xs : \LL(A)$
    is given by $\fib_{\ssection}(\xs) \defeq \dsum{ys : \MM(A)}{(\ssection(ys) = \xs)}$.
    %
    The image of $\ssection$ is given by $\im{\ssection} \defeq \dsum{\xs : \LL(A)}{\Trunc[-1]{\fib_{\ssection}(\xs)}}$.
    %
    Simplifying, we say that $\xs:\LL(A)$ is ``in the image of $\ssection$'', or, $\xs \in \im{\ssection}$,
    if there merely exists a $\ys:\MM(A)$ such that $\ssection(\ys) = \xs$.
\end{definition}
%
If $\ssection$ \emph{were} a sort function, the image of $\ssection$ would be the set of $\ssection$-``sorted'' lists,
therefore $\inimage{\xs}$ would imply $\xs$ is a correctly $\ssection$-``sorted'' list.
%
First, we note that the 2-element case is correct.
%
\begin{proposition}[\alink{proposition}{}]
    \label{sort:sort-to-order}
    $x \leqs y$ \; iff \; $\inimage{[x, y]}$.
\end{proposition}
%
\noindent Then, we state the first axiom on $\ssection$.
\begin{definition}[\alink{definition}{$\isheadleast$}]
    \label{sort:head-least}
    A section $\ssection$ satisfies $\isheadleast$ iff for all $x, y, \xs$:
    \[
        y \in x \cons \xs \; \land \; \inimage{x \cons \xs} \; \to \; \inimage{[x, y]}
        \enspace.
    \]
\end{definition}
\noindent
We use the definition of list membership from~\cref{def:membership}.
The $\in$ symbol is intentionally overloaded
to make the axiom look like a logical ``cut'' rule.
Informally, it says that the head of an $\ssection$-``sorted'' list
is the least element of the list.

\begin{proposition}[\alink{proposition}{}]
    \label{prop:order-to-sort-head-least}
    If $A$ has a total order $\leq$,
    insertion sort defined using $\leq$ satisfies $\isheadleast$.
\end{proposition}

\begin{proposition}[\alink{proposition}{}]
    \label{sort:trans}
    If $\ssection$ satisfies $\isheadleast$, $\leqs$ is transitive.
\end{proposition}
\begin{proof}
    Given $x \leqs y$ and $y \leqs z$, we want to show $x \leqs z$.
    Consider the 3-element bag $\bag{x,y,z} : \MM(A)$.
    %
    Let $u$ be $\term{least}(\bag{x,y,z})$,
    by~\cref{sort:head-least} and~\cref{sort:sort-to-order},
    we have $u \leqs x \land u \leqs y \land u \leqs z$.
    %
    Since $u \in \bag{x,y,z}$, $u$ must be one of the elements.
    %
    If $u = x$ we have $x \leqs z$.
    If $u = y$ we have $y \leqs x$,
    and since $x \leqs y$ and $y \leqs z$ by assumption,
    we have $x = y$ by antisymmetry, and then we have $x \leqs z$ by substitution.
    Finally, if $u = z$, we have $z \leqs y$, and since $y \leqs z$ and $x \leqs y$ by assumption,
    we have $z = y$ by antisymmetry, and then we have $x \leqs z$ by substitution.
\end{proof}

\subsubsection{Embedding orders into sections}
\label{sec:sort-embedding}

Following from \cref{sort:almost-total,sort:trans},
and \cref{prop:order-to-sort-head-least},
we have shown that a section $\ssection$ that satisfies $\isheadleast$ produces a total order
$x \leqs y \defeq \term{least}(\bag{x, y}) \id \inr(x)$,
and a total order $\leq$ on the carrier set produces a section satisfying $\isheadleast$,
constructed by sorting with $\leq$.
%
This constitutes an embedding of decidable total orders into sections satisfying $\isheadleast$.

\begin{proposition}[\alink{proposition}{}]
    \label{sort:o2s2o}
    Assume $A$ has a decidable total order $\leq$, we can construct a section $\ssection$ that
    satisfies $\isheadleast$, such that $\leqs$ constructed from $\ssection$ is equivalent
    to $\leq$.
\end{proposition}
\begin{proof}
    By the insertion sort algorithm parameterized by $\leq$,
    it holds that $\inimage{[x, y]}$ iff $x \leq y$.
    By~\cref{sort:sort-to-order}, we have $x \leqs y$ iff $x \leq y$.
    We now have a total order $x \leqs y$ equivalent to $x \leq y$.
\end{proof}

\subsubsection{Equivalence of order and sections}

We want to upgrade the embedding to an isomorphism, and it
remains to show that we can turn a section satisfying $\isheadleast$ to a total order $\leqs$,
then construct the \emph{same} section back from $\leqs$.
Unfortunately, $\isheadleast$ is not enough to guarantee this.
\begin{proposition}
    \label{prop:counterexample-equivalence}
    Assume $A$ is a set with different elements, i.e. $\exists x, y: A.\,x \neq y$,
    we cannot construct a full equivalence between sections that satisfy $\isheadleast$
    and total orders on $A$.
\end{proposition}
\begin{proof}
    We give a counter-example of $\ssection$ that satisfy $\isheadleast$ but is not a sort function.
    Consider the insertion sort function $\term{sort} : \MM(\Nat) \to \LL(\Nat)$
    parameterized by $\leq$:
    \begin{align*}
        \term{reverseTail}([])          & = []                                   \\
        \term{reverseTail}(x \cons \xs) & = x \cons \term{reverse}(\xs)          \\
        s(\xs)                          & = \term{reverseTail}(\term{sort}(\xs)) \\
        s(\bag{2,3,1,4})                & = [1,4,3,2]                            \\
        s(\bag{2,3,1})                  & = [1, 3, 2]                            \\
        s(\bag{2,3})                    & = [2, 3]                               \\
    \end{align*}
    By~\cref{sort:o2s2o} we can use $\term{sort}$ to construct $\leqs$ which would be
    equivalent to $\leq$. However, the $\leqs$ constructed by $\ssection$ would also be equivalent
    to $\leq$. This is because $\ssection$ sorts 2-element list correctly, despite $s \neq \term{sort}$.
    Since two different sections satisfying $\isheadleast$ maps to the same total order,
    there cannot be a full equivalence.
\end{proof}

Therefore, we need to introduce a second axiom of sorting.

\begin{definition}[\alink{definition}{$\istailsort$}]
    \label{def:tail-sort}
    A section $\ssection$ satisfies $\istailsort$ iff
    for all $x, \xs$,
    \[
        \inimage{x \cons \xs} \to \inimage{\xs}
    \]
\end{definition}

This says that $\ssection$-``sorted'' lists are downwards-closed under cons-ing, that is,
the tail of an $\ssection$-``sorted'' list is also $\ssection$-``sorted''.
%
To prove the correctness of our axioms,
first we need to show that a section $\ssection$ satisfying
$\isheadleast$ and $\istailsort$ is equal to insertion sort parameterized by
the $\leqs$ constructed from $\ssection$.
%
In fact, the axioms we have introduced are equivalent to the standard inductive characterization of sorted lists,
found in textbooks, such as in~\cite{appelVerifiedFunctionalAlgorithms2023}.

\begin{code}
data IsSorted ($\leq$ : A -> A -> UU) : List A -> UU where
  sorted-nil : IsSorted []
  sorted-$\eta$ : forall x -> IsSorted [ x ]
  sorted-cons : forall x y zs -> x $\leq$ y
     -> IsSorted (y cons zs) -> IsSorted (x cons y cons zs)
\end{code}
Note that $\term{IsSorted}_{\leq}(\xs)$ is a proposition for every $\xs$,
and forces the list $\xs$ to be permuted in a unique way.
\begin{lemma}[\alink{lemma}{}]
    Given a total order $\leq$, for any $\xs, \ys : \LL(A)$,
    $q(\xs) = q(\ys) \land \term{IsSorted}_{\leq}(\xs) \land \term{IsSorted}_{\leq}(\ys) \to \xs = \ys$.
\end{lemma}

Insertion sort by $\leq$ always produces lists that satisfy $\term{IsSorted}_{\leq}$.
Functions that also produce lists satisfying $\term{IsSorted}_{\leq}$ are equal to insertion sort
by function extensionality.

\begin{proposition}[\alink{proposition}{}]
    \label{sort:sort-uniq}
    Given a total order $\leq$,
    if a section $\ssection$ always produces sorted list, i.e. $\forall \xs.\,\term{IsSorted}_{\leq}(s(\xs))$,
    $\ssection$ is equal to insertion sort by $\leq$.
\end{proposition}
\noindent
Finally, this gives us correctness of our axioms.

\begin{proposition}[\alink{proposition}{}]
    \label{sort:well-behave-sorts}
    Given a section $\ssection$ that satisfies $\isheadleast$ and $\istailsort$,
    and $\leqs$ the order derived from $\ssection$, then for all $\xs : \MM(A)$,
    it holds that $\term{IsSorted}_{\leqs}(s(\xs))$.
    %
    Equivalently, for all lists $\xs : \LL(A)$,
    it holds that
    $\xs \in \im{s}$ iff $\term{IsSorted}_{\leqs}(\xs)$.
\end{proposition}
\begin{proof}
    We inspect the length of $\xs : \MM(A)$.
    For lengths 0 and 1, this holds trivially.
    Otherwise, we proceed by induction:
    given a $\xs : \MM(A)$ of length $\geq 2$, let $s(\xs) = x \cons y \cons \ys$.
    We need to show
    $x \leqs y \land \term{IsSorted}_{\leqs}(y \cons \ys)$ to construct
    $\term{IsSorted}_{\leqs}(x \cons y \cons \ys)$.
    By $\isheadleast$, we have $x \leqs y$, and by $\istailsort$, we
    inductively prove $\term{IsSorted}_{\leqs}(y \cons \ys)$.
\end{proof}

\begin{lemma}[\alink{lemma}{}]
    \label{sort:s2o2s}
    Given a decidable total order $\leq$ on $A$, we can construct
    a section $t_\leq$ satisfying $\isheadleast$ and $\istailsort$,
    such that, for the order $\leqs$ derived from $\ssection$,
    we have $t_{\leqs} = \ssection$.
\end{lemma}
\begin{proof}
    From $\ssection$ we can construct a decidable total order $\leqs$ since $\ssection$ satisfies
    $\isheadleast$ and $A$ has decidable equality by assumption.
    We construct $t_{\leqs}$ as insertion sort
    parameterized by $\leqs$ constructed from $\ssection$.
    By ~\cref{sort:sort-uniq} and ~\cref{sort:well-behave-sorts}, $s = t_{\leqs}$.
\end{proof}

\noindent
We can now state and prove our main theorem.
\begin{definition}[\alink{definition}{Sorting function}]
    \leavevmode
    A sorting function is a section $\ssection : \MM(A) \to \LL(A)$ to
    the canonical surjection $\quotient : \LL(A) \twoheadrightarrow \MM(A)$ satisfying two axioms:
    \begin{itemize}[leftmargin=*]
        \item $\isheadleast$:
              \(\,
              \inimage{x \cons \xs} \land y \in x \cons \xs \to \inimage{[x, y]}
              \),
        \item $\istailsort$:
              \(\,
              \inimage{x \cons \xs} \to \inimage{\xs}
              \).
    \end{itemize}
\end{definition}
\begin{theorem}[\alink{theorem}{}]
    \label{sort:main}
    Let $\term{DecTotOrd}(A)$ be the set of decidable total orders on $A$,
    $\term{Sort}(A)$ be the set of sorting functions with carrier set $A$,
    and $\term{Discrete}(A)$ be a predicate which states $A$ has decidable equality.
    There is a map $o2s \colon \term{DecTotOrd}(A) \to \term{Sort}(A) \times \term{Discrete}(A)$,
    which is an equivalence.
\end{theorem}
\begin{proof}
    $o2s$ is constructed by parameterizing insertion sort with $\leq$.
    By~\cref{prop:decidable-total-order}, $A$ is $\term{Discrete}$.
    %
    The inverse $s2o(s)$ is constructed by~\cref{def:least}, which produces
    a total order by~\cref{sort:almost-total,sort:trans},
    and a decidable total order by $\term{Discrete}(A)$.
    %
    By~\cref{sort:o2s2o} we have $s2o \comp o2s \id \idfunc$,
    and by~\cref{sort:s2o2s} we have $o2s \comp s2o \id \idfunc$,
    giving an isomorphism, hence an equivalence.
\end{proof}

\begin{corollary}[\alink{corollary}{}]
    \label{sort:linear-order}
    Let $\term{DecLinOrd}(A)$ be the set of decidable strict total orders on $A$,
    There is a map $l2s \colon \term{DecLinOrd}(A) \to \term{Sort}(A) \times \term{Discrete}(A)$,
    which is an equivalence.
\end{corollary}
\begin{proof}
    By~\cref{prop:decidable-total-order}, we have an equivalence
    $\term{DecLinOrd}(A) \simeq \term{DecTotOrd}(A)$, and the result follows from~\cref{sort:main}
    by equivalence transitivity.
\end{proof}
%
The sorting axioms we have come up with are abstract properties of functions.
%
As a sanity check, we can verify that the colloquial correctness specification of a sorting function (starting from a
total order) matches our axioms. We use the correctness criterion developed in~\cite{alexandruIntrinsicallyCorrectSorting2023}.
%
\begin{proposition}[\alink{proposition}{}]
    \label{prop:sort-correctness}
    Assume a decidable total order $\leq$ on $A$.
    %
    A sorting algorithm is a map $\term{sort} : {\LL(A) \to \OLL(A)}$,
    that turns lists into ordered lists,
    where $\OLL(A)$ is defined as $\dsum{\xs : \LL(A)}{\term{IsSorted}_{\leq}(\xs)}$,
    such that:
    % https://q.uiver.app/#q=WzAsMyxbMCwwLCJcXExMKEEpIl0sWzIsMCwiXFxPTEwoQSkiXSxbMSwxLCJcXE1NKEEpIl0sWzAsMSwiXFx0ZXJte3NvcnR9Il0sWzAsMiwicSIsMl0sWzEsMiwicSBcXGNvbXAgXFxwaV8xIl1d
    \[\begin{tikzcd}
            {\LL(A)} && {\OLL(A)} \\
            & {\MM(A)}
            \arrow["{\term{sort}}", from=1-1, to=1-3]
            \arrow["q"', from=1-1, to=2-2]
            \arrow["{q \comp \pi_1}", from=1-3, to=2-2]
        \end{tikzcd}\]
    Sorting functions give sorting algorithms.
\end{proposition}
\begin{proof}
    We construct our section $\ssection:\MM(A) \to \LL(A)$,
    and define $\term{sort} \defeq \ssection \comp \quotient$,
    which produces ordered lists by~\cref{sort:well-behave-sorts}.
\end{proof}
 %% 3 pages
