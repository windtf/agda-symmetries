% !TEX root = types25pp-sort.tex

\section{Introduction}
\label{sec:introduction}

Consider a puzzle about sorting,
inspired by Dijkstra's Dutch National Flag problem~\cite[Ch.14]{dijkstraDisciplineProgramming1997}.
Suppose there are balls of three colors,
corresponding to the colors of the Dutch flag: red, white, and blue.
\[
  \{
  \tikz[anchor=base, baseline]{%
    \foreach \x/\color in {1/red,2/white,3/blue} {
        \node[circle,draw,fill=\color,line width=1pt] at (\x,0) {\phantom{\tiny\x}};
        \ifthenelse{\NOT 1 = \x}{\node at ({\x-0.5},0) {,};}{}
      }
  }
  \}
\]
Given an unordered list (bag) of such balls, how many ways can you sort them into the Dutch flag?
\[
  \bag{
    \tikz[anchor=base, baseline]{%
      \foreach \x/\color in {1/red,2/red,3/blue,4/white,5/blue,6/red,7/white,8/blue} {
          \node[circle,draw,fill=\color,line width=1pt] at (\x,0) {\phantom{\tiny\x}};
          \ifthenelse{\NOT 1 = \x}{\node at ({\x-0.5},0) {,};}{}
        }
    }
  }
\]
Obviously there is only one way, decided by the order the colors appear in the Dutch flag:
$\term{red} < \term{white} < \term{blue}$.
\[
  [
      \tikz[anchor=base, baseline]{%
        \foreach \x/\color in {1/red,2/red,3/red,4/white,5/white,6/blue,7/blue,8/blue} {
            \node[circle,draw,fill=\color,line width=1pt] at (\x,0) {\phantom{\tiny\x}};
            \ifthenelse{\NOT 1 = \x}{\node at ({\x-0.5},0) {,};}{}
          }
      }
    ]
\]
What if we are avid enjoyers of vexillology who also want to consider other flags?
We might ask: how many ways can we sort our unordered list of balls?
We know there are exactly $3! = 6$ permutations of
$\{\term{red}, \term{white}, \term{blue}\}$, so there are 6 possible orderings we can define.
In fact, there are exactly 6 such categories of tricolor flags
(see~\href{https://en.wikipedia.org/wiki/List_of_flags_with_blue,_red,_and_white_stripes#Triband}{Wikipedia}).
We have no allegiance to any of the countries presented by the flags, hypothetical or otherwise --
this is purely a matter of combinatorics.
\begin{center}
  \foreach \colorA/\colorB/\colorC in {red/white/blue, red/blue/white, white/red/blue, white/blue/red, blue/red/white, blue/white/red}{
      \begin{tikzpicture}[scale=0.5]
        \begin{flagdescription}{3/4}
          \hstripesIII{\colorA}{\colorB}{\colorC}
          \framecode{}
        \end{flagdescription}
      \end{tikzpicture}
    }
\end{center}
We posit that, because there are exactly 6 orderings, we can only define 6 \emph{extensionally correct} sorting functions.
%
Formally, there is a bijection between the set of orderings on a carrier set $A$ and the set of correct sorting
functions on lists of $A$.
%
In fact, a sorting function can be correctly axiomatized just from this point of view, which is our main contribution.

\paragraph*{Outline and Contributions}
% The paper is organized as follows:
\begin{myitemize}
  \item In~\cref{sec:universal-algebra}, we describe a formalisation of universal algebra developed from the point of view
        of functorial signatures, the definition and universal property of free algebras,
        and algebras satisfying an equational theory.
  \item In~\cref{sec:monoids}, we give constructions of free monoids, and proofs of their universal property.
        Following this, in~\cref{sec:commutative-monoids}, we add symmetry to each representation of free monoids,
        and extend the proofs of universal property from free monoids to free commutative monoids.
        These are standard constructions that we formalise conceptually by formal combinatorics.
  \item In \cref{sec:application}, we build on the previous constructions and study sorting functions.
        The main result connects total orders, sorting, and symmetry,
        by proving an equivalence between decidable total orders on a carrier set $A$,
        and correct sorting functions on lists of $A$.
  \item Finally, \cref{sec:discussion} discusses related and future work.
\end{myitemize}
%
The three main parts of the paper may be read independently.
%
Readers interested in the formalisation of universal algebra may start from~\cref{sec:universal-algebra}.
%
Readers interested in the constructions of free monoids and free commutative monoids may skip ahead
to~\cref{sec:monoids,sec:commutative-monoids}.
%
If the reader already believes in the existence of free algebras for monoids and commutative monoids,
they can directly skip to the application section on sorting, in~\cref{sec:application}.
%
Although the formalisation is a contribution in itself, the purpose of the paper is not to directly discuss the
formalisation, but to present the results in un-formalised form (in type-theoretic foundations), so the ideas are
accessible to a wider audience.
%
\ifarxiv{}{The extended version of this paper including additional details and proofs is available on Zenodo~\cite{extendedVersion}.}
%
All the results of this paper are formalised in Cubical Agda, and the formalisation artifact is available at~\cite{choudhuryAgdasymmetries2025}.