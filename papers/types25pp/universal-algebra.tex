% !TEX root = types25pp-sort.tex

\section{Universal Algebra}
\label{sec:universal-algebra}

We first develop some basic notions from universal algebra and equational
logic~\cite{birkhoffStructureAbstractAlgebras1935},
which gives us a vocabulary and framework to express our results in.
%
The point of view we take is the standard category-theoretic approach to universal algebra, which predates the Lawvere
theory or abstract clone point of view.
%
We keep a running example of monoids in mind, while explaining and defining the abstract concepts.
% In the language of universal algebra, such a structure is formalized by giving a signature of operations, and a
% structure being a carrier set with functions that interpret these operations. We describe such a framework for universal
% algebra in HoTT, as follows.

\subsection{Algebras}
\label{sec:universal-algebra:algebras}

\begin{definition}[\alink{definition}{Signature}]
    \label{algebra:signature}
    \label{def:signature}
    A signature, denoted $\sig$, is a (dependent) pair consisting of
    a set of operations, $\op\colon \Set$, and
    an arity function for each operation, $\ar\colon \op \to \Set$.
\end{definition}

\begin{example}
    A premonoid is a set with an identity element (a nullary operation), and a binary multiplication operation,
    with signature $\sigMon \defeq
        (\Fin[2],\lambda \{0 \mapsto \Fin[0] ; 1 \mapsto \Fin[2] \})$.
    %
    % Informally, the set of operations is the two-element set $\{e,\mult\}$, which is written as $\Fin[2]$,
    % and the arity function picks out a (finite) set denoting the (finite) arity of each operation.
    % %
    % Of course, this is an example of a finitary signature,
    % but in general the arity function can be any (not necessarily finite) set.
\end{example}

\noindent
Every signature $\sig$ induces a signature functor $\Sig$ on $\Set$.
\begin{definition}[\alink{definition}{Signature functor} $\Sig \colon \Set \to \Set$]
    \label{def:signature-functor}
    \[
        X \mapsto \dsum{o:\op}{X^{\ar(o)}}
        \qquad
        X \xto{f} Y \mapsto
        \dsum{o:\op}{X^{\ar(o)}}
        \xto{(o, \blank \comp f)}
        \dsum{o:\op}{Y^{\ar(o)}}
    \]
\end{definition}

\begin{example}
    The signature functor for monoids, $\SigMon$, assigns to a carrier set $X$,
    the set of inputs for each operation.
    %
    Expanding the dependent product on $\Fin[2]$, we obtain a coproduct of sets:
    $\SigMon(X) \eqv X^{\Fin[0]} + X^{\Fin[2]} \eqv \unitt + X \times X$.
\end{example}
%
A $\sig$-structure is given by a carrier set, with functions interpreting each operation symbol.
%
The signature functor applied to a carrier set gives the inputs to each operation,
and the output is simply a map back to the carrier set.
%
These two pieces of data constitute an algebra for the $\Sig$ functor.
%
We write $\str{X}$ for a $\sig$-structure with carrier set $X$, following the model-theoretic notational convention.

\begin{definition}[\alink{definition}{Structure}]
    \label{algebra:struct}
    A $\sig$-structure $\str{X}$ is an $\Sig$-algebra, that is, a pair consisting of
    a carrier set $X$, and
    an algebra map $\alpha_{X}\colon \Sig(X) \to X$.
\end{definition}

\begin{example}
    Concretely, an $\SigMon$-algebra has the type
    \[
        \alpha_{X} : \SigMon(X) \to X
        \quad\eqv\quad
        (\unitt + X \times X) \to X
        \quad\eqv\quad
        (\unitt \to X) \times (X \times X \to X)
    \]
    which is the pair of functions interpreting the two operations.
    Natural numbers $\Nat$ with $(0, +)$ or $(1, \times)$ are examples of monoid structures.
\end{example}

\begin{definition}[\alink{definition}{$\sig$-Homomorphism}]
    A homomorphism between $\sig$-structures $\str{X}$ and $\str{Y}$ is a morphism of $\Sig$-algebras,
    that is, a map $f : X \to Y$ making the following diagram commute:
    \[
        \begin{tikzcd}
            \Sig(X) \arrow[r, "\alpha_{X}"] \arrow[d, "\Sig(f)"']
            & X \arrow[d, "f"] \\
            \Sig(Y) \arrow[r, "\alpha_{Y}"']
            & Y
        \end{tikzcd}
    \]
\end{definition}

\begin{example}
    Given two monoids \(\str{X}\) and \(\str{Y}\), the top half denotes:
    \({\unitt + (X \times X) \xto{\alpha_{X}} X \xto{f} Y}\), which applies \(f\) to the output of each operation,
    and the bottom half denotes the map:
    \({\unitt + (X \times X) \xto{\SigMon(f)} \unitt + (Y \times Y) \xto{\alpha_{Y}} Y}\).
    % Commutativity of the diagram says that these two maps are equal.
    %
    In other words, a homomorphism between $X$ and $Y$ is a map $f$ on the carrier sets that commutes with the
    interpretation of the monoid operations, or simply, preserves the monoid structure.
\end{example}
%
For a fixed signature $\sig$,
the category of $\Sig$-algebras and their morphisms form a category of algebras,
written $\SigAlg$, or simply, $\sigAlg$,
given by the obvious definitions of identity and composition of the underlying functions.

\subsection{Free Algebras}
\label{sec:universal-algebra:free-algebras}

The concrete category $\sigAlg$ of structured sets and structure-preserving maps
admits a forgetful functor $U : \sigAlg \to \Set$.
%
In our notation, $U(\str{X})$ is simply $X$, a fact we exploit to simplify our notation, and formalization.
%
The left adjoint to this forgetful functor is the free algebra construction,
also known as the term algebra (or the \emph{absolutely free} algebra without equations).
%
We rephrase this in more concrete terms.

\begin{definition}[\alink{definition}{Free Algebras}]
    \label{def:free-algebras}
    A free $\sig$-algebra construction consists of the following:
    \begin{itemize}
        \item a set $F(X)$, for every set $X$,
        \item a $\sig$-structure on $F(X)$, written as $\str{F}(X)$,
        \item a universal map $\eta_X : X \to F(X)$, for every $X$, such that,
        \item for any $\sig$-algebra $\str{Y}$, the operation
              assigning to each homomorphism $f : \str{X} \to \str{Y}$,
              the map ${f \comp \eta_X : X \to Y}$ (or, post-composition with $\eta_X$),
              is an equivalence.
    \end{itemize}
\end{definition}
In other words,
we ask for a bijection between the set of homomorphisms out of the free algebra to any other algebra,
and the set of functions from the carrier set of the free algebra to the carrier set of the other algebra.
%
There should be no more data in homomorphisms out of the free algebra than there is in functions out of
the carrier set, which is the property of \emph{freeness}.
%
The inverse operation to post-composing with $\eta_X$ extends a function to a homomorphism.
%
Note that the free algebra on the empty set~\(\str{F}(\emptyt)\) is inhabited by all the constant symbols in the signature.

\begin{definition}[\alink{definition}{Universal extension}]
    \label{def:universal-extension}
    The universal extension of a function $f : X \to Y$ to a homomorphism out of the free $\sig$-algebra on $X$ is written
    as $\ext{f} : \str{F}(X) \to \str{Y}$.
    %
    It satisfies the identities: $\ext{f} \comp \eta_X = f$, $\ext{\eta_{X}} = \idfunc_{\str{F}(X)}$,
    and $\ext{(\ext{g} \comp f)} = \ext{g} \comp \ext{f}$.
\end{definition}
\noindent
Free algebra constructions are canonically equivalent.
\begin{propositionrep}[\alink{proposition}{}]
    \label{lem:free-algebras-unique}
    If $\str{F}(X)$ and $\str{G}(X)$ are free $\sig$-algebras on $X$,
    %
    then $\str{F}(X) \eqv \str{G}(X)$ in $\sigAlg$, naturally in $X$, via the canonical extensions.
\end{propositionrep}
\begin{proof}
    By extending $\eta_X$ for each free construction,
    we have maps in each direction:
    ${\ext{G\fdot\eta_{X}} : \str{F}(X) \to \str{G}(X)}$, and vice versa.
    %
    Finally, using~\cref{def:universal-extension}, we have
    \(
    {\ext{F\fdot\eta_{X}} \comp \ext{G\fdot\eta_{X}}} =
    {\ext{(\ext{F\fdot\eta_{X}} \comp G\fdot\eta_{X})}} =
        {\ext{F\fdot\eta_{X}}} =
        {\idfunc_{\str{F}(X)}}
    \).
    % \vc{Mention SIP.}
\end{proof}
\begin{toappendix}
    \begin{proposition}
        The free algebra construction gives a monad on $\Set$.
    \end{proposition}
    \begin{proof}
        $F$ is an endofunctor on $\Set$ where the action on functions is given by
        $X \xto{f} Y \mapsto F(X) \xto{\ext{(\eta_{Y} \comp f)}} F(Y)$.
        %
        The monad unit is given by $\eta$,
        and multiplication given by $\mu_{X} \defeq F(F(X)) \xto{\ext{\idfunc_{F(X)}}} F(X)$.
    \end{proof}

    \begin{proposition}
        \label{prop:free-algebra-colimits}
        \leavevmode
        The following properties of free algebras on 0, 1, and coproducts hold:
        \begin{itemize}
            \item $\sigAlg(\str{F}(\emptyt), \str{X})$ is contractible,
            \item if $\sig$ has one constant symbol, then $\str{F}(\emptyt)$ is contractible,
            \item the type of algebra structures on $\unitt$ is contractible,
            \item $\str{F}(X + Y)$ is the coproduct of $\str{F}(X)$ and $\str{F}(Y)$ in $\sigAlg$:
                  $\sigAlg(\str{F}(X + Y), \str{Z}) \eqv \sigAlg(\str{F}(X), \str{Z}) \times \sigAlg(\str{F}(Y), \str{Z})$.
        \end{itemize}
    \end{proposition}
    \begin{proof}
        $F$ being a left adjoint, preserves coproducts.
        %
        This makes $\str{F}(\emptyt)$ initial in $\sigAlg$.
        %
        $\str{F}(\unitt) \to \unitt$ is contractible because $\unitt$ is terminal in $\Set$.
    \end{proof}
\end{toappendix}
We have discussed specifications of free algebras, but not actually given a construction.
%
In type theory, \emph{free} constructions are often given by inductive types,
where the constructors are the pieces of data that freely generate the structure,
and the type-theoretic induction principle enforces the category-theoretic universal property.

\begin{definition}[\alink{definition}{Construction of Free $\sig$-algebras}]
    \label{algebra:tree}
    \label{def:free-algebra-construction}
    The free $\sig$-algebra on a type $X$ is given by the inductive type $\type{Tree}(X)$,
    generated by two constructors:
    \begin{align*}
        \term{leaf} & \colon X \to \type{Tree}(X)                      \\
        \term{node} & \colon F_\sig(\type{Tree}(X)) \to \type{Tree}(X)
    \end{align*}
\end{definition}
The constructors \inline{leaf} and \inline{node} are
the generators for the universal map, and the algebra map, respectively.
%
Together, they describe the type of abstract syntax trees for terms in the signature $\sig$
-- the leaves are the free variables, and the nodes are the branching operations of the tree,
marked by the operations in $\sig$.
\begin{toappendix}
    \begin{example}
        A tree for $\sigMon$ with the carrier set $\Nat$ would look like:
        \begin{center}
            \scalebox{0.7}{
                % https://q.uiver.app/#q=WzAsNixbMiwyLCIrIl0sWzIsMywiNCJdLFszLDEsIisiXSxbMiwwLCIxIl0sWzQsMCwiMSJdLFswLDAsIjIiXSxbMCwxXSxbMiwwXSxbMywyXSxbNCwyXSxbNSwwXV0=
                \begin{tikzcd}[ampersand replacement=\&,cramped]
                    2 \&\& 1 \&\& 1 \\
                    \&\&\& {+} \\
                    \&\& {+} \\
                    \&\&
                    \arrow[no head, from=3-3, to=4-3]
                    \arrow[from=2-4, to=3-3]
                    \arrow[from=1-3, to=2-4]
                    \arrow[from=1-5, to=2-4]
                    \arrow[from=1-1, to=3-3]
                \end{tikzcd}
            }
            \hspace{2em}
            % https://q.uiver.app/#q=WzAsMixbMCwwLCJlIl0sWzAsMiwiMCJdLFswLDFdXQ==
            \begin{tikzcd}[ampersand replacement=\&,cramped]
                e \\
                \\
                \arrow[no head, from=1-1, to=3-1]
            \end{tikzcd}
        \end{center}
    \end{example}
\end{toappendix}
%
\begin{proposition}[\alink{proposition}{}]
    \label{prop:free-algebra-construction-is}
    $(\type{Tree}(X), \term{leaf})$ is the free $\sig$-algebra on~$X$. %% for any set $X$.
\end{proposition}

\subsection{Equations}
\label{sec:universal-algebra:equations}

The algebraic framework described so far only captures operations, not equations.
%
These algebras are \emph{lawless} (or \emph{wild} or \emph{absolutely free}) --
$\SigMon$-algebras are premonoids rather than monoids,
and $\str{F}_{\sigMon}$-algebras are free premonoids, not free monoids,
since they are missing the unit and associativity laws.
\begin{toappendix}
    For example, by associativity, these two trees of $(\Nat, +)$ should be identified as equal.
    \begin{center}
        \scalebox{0.7}{
            % https://q.uiver.app/#q=WzAsNixbMiwyLCIrIl0sWzIsMywiNCJdLFszLDEsIisiXSxbMiwwLCIxIl0sWzQsMCwiMSJdLFswLDAsIjIiXSxbMCwxXSxbMiwwXSxbMywyXSxbNCwyXSxbNSwwXV0=
            \begin{tikzcd}[ampersand replacement=\&,cramped]
                2 \&\& 1 \&\& 1 \\
                \&\&\& {+} \\
                \&\& {+} \\
                \&\&
                \arrow[no head, from=3-3, to=4-3]
                \arrow[from=2-4, to=3-3]
                \arrow[from=1-3, to=2-4]
                \arrow[from=1-5, to=2-4]
                \arrow[from=1-1, to=3-3]
            \end{tikzcd}
            \hspace{2em}
            % https://q.uiver.app/#q=WzAsNixbMiwyLCIrIl0sWzIsM10sWzIsMCwiMSJdLFs0LDAsIjEiXSxbMCwwLCIyIl0sWzEsMSwiKyJdLFswLDEsIiIsMCx7InN0eWxlIjp7ImhlYWQiOnsibmFtZSI6Im5vbmUifX19XSxbMywwXSxbNCw1XSxbMiw1XSxbNSwwXV0=
            \begin{tikzcd}[ampersand replacement=\&,cramped]
                2 \&\& 1 \&\& 1 \\
                \& {+} \\
                \&\& {+} \\
                \&\& {}
                \arrow[no head, from=3-3, to=4-3]
                \arrow[from=1-5, to=3-3]
                \arrow[from=1-1, to=2-2]
                \arrow[from=1-3, to=2-2]
                \arrow[from=2-2, to=3-3]
            \end{tikzcd}
        }
    \end{center}
\end{toappendix}
%
To impose equations on the generated abstract syntax trees, we adopt the point of view of equational logic.
%
\begin{definition}[\alink{definition}{Equational Signature}]
    \label{def:equational-signature}
    An equational signature, denoted $\eqsig$, is a (dependent) pair consisting of:
    a set of names for equations, $\eqop : \Set$, and
    an arity of free variables for each equation, $\eqfv : \eqop \to \Set$.
\end{definition}
%
\begin{example}
    The equational signature for monoids $\eqsigMon$ is:
    $(\Fin[3],\lambda \{0 \mapsto \Fin[1] ; 1 \mapsto \Fin[1] ; 2 \mapsto \Fin[3] \})$.
    %
    The three equations are the left and right unit laws, and the associativity law --
    a 3-element set of names $\{ \term{unitl}, \term{unitr}, \term{assoc} \}$.
    %
    The two unit laws use one free variable,
    and the associativity law uses three free variables.
\end{example}
%
Similar to the term signature functor in~\cref{def:signature-functor}, this produces an equational signature functor on $\Set$.
\begin{definition}[\alink{definition}{Eq. Signature Functor $\EqSig \colon \Set \to \Set$}]
    \label{def:equational-signature-functor}
    \begin{gather*}
        X \mapsto \dsum{e:\eqop}{X^{\eqfv(e)}}
        \qquad
        X \xto{f} Y \mapsto
        \dsum{e:\eqop}{X^{\ar(e)}}
        \xto{(e, \blank \comp f)}
        \dsum{e:\eqop}{Y^{\ar(e)}}
    \end{gather*}
\end{definition}
%
To build equations out of this,
we use the $\sig$-operations and construct trees for the left and right-hand sides of each equation using the
free variables available.
%
\begin{definition}[\alink{definition}{System of Equations}]
    A system of equations over a signature $(\sig,\eqsig)$ is a pair of natural transformations:
    \[
        \eqleft, \eqright \colon \EqSig \natto \str{F}_{\sig} \enspace.
    \]
    For any set (of variables) $V$,
    this gives a pair of functions $\eqleft_{V}, \eqright_{V}\colon \EqSig(V) \to \str{F}_{\sig}(V)$,
    and naturality ensures correctness of renaming.
\end{definition}
\begin{toappendix}
    \begin{example}
        Given $x\colon V$, $\eqleft_{V}(\term{unitl}, (x)),\,\eqright_{V}(\term{unitl}, (x))$
        are defined as:
        \begin{center}
            % https://q.uiver.app/#q=WzAsNCxbMCwwLCIwIl0sWzEsMSwiXFxidWxsZXQiXSxbMiwwLCJlIl0sWzEsMl0sWzAsMV0sWzIsMV0sWzEsMywiIiwyLHsic3R5bGUiOnsiaGVhZCI6eyJuYW1lIjoibm9uZSJ9fX1dXQ==
            \begin{tikzcd}[ampersand replacement=\&,cramped]
                e \&\& x \\
                \& \mult \\
                \& {}
                \arrow[from=1-1, to=2-2]
                \arrow[from=1-3, to=2-2]
                \arrow[no head, from=2-2, to=3-2]
            \end{tikzcd}
            \hspace{2em}
            \begin{tikzcd}[ampersand replacement=\&,cramped]
                x \\
                \\
                \arrow[no head, from=1-1, to=3-1]
            \end{tikzcd}
        \end{center}
        %
        Given $x\colon V$, $\eqleft_{V}(\term{unitr}, (x)),\,\eqright_{V}(\term{unitr}, (x))$
        are defined as:
        \begin{center}
            % https://q.uiver.app/#q=WzAsNCxbMCwwLCIwIl0sWzEsMSwiXFxidWxsZXQiXSxbMiwwLCJlIl0sWzEsMl0sWzAsMV0sWzIsMV0sWzEsMywiIiwyLHsic3R5bGUiOnsiaGVhZCI6eyJuYW1lIjoibm9uZSJ9fX1dXQ==
            \begin{tikzcd}[ampersand replacement=\&,cramped]
                x \&\& e \\
                \& \mult \\
                \& {}
                \arrow[from=1-1, to=2-2]
                \arrow[from=1-3, to=2-2]
                \arrow[no head, from=2-2, to=3-2]
            \end{tikzcd}
            \hspace{2em}
            \begin{tikzcd}[ampersand replacement=\&,cramped]
                x \\
                \\
                \arrow[no head, from=1-1, to=3-1]
            \end{tikzcd}
        \end{center}
        %
        Given $x, y, z : V$, $\eqleft_{V}(\term{assocr}, (x,y,z)),\,\eqright_{V}(\term{assocr}, (x,y,z))$
        are defined as:
        \begin{center}
            \scalebox{0.7}{
                % https://q.uiver.app/#q=WzAsNixbMiwyLCIrIl0sWzIsMywiNCJdLFszLDEsIisiXSxbMiwwLCIxIl0sWzQsMCwiMSJdLFswLDAsIjIiXSxbMCwxXSxbMiwwXSxbMywyXSxbNCwyXSxbNSwwXV0=
                \begin{tikzcd}[ampersand replacement=\&,cramped]
                    x \&\& y \&\& z \\
                    \&\&\& {\mult} \\
                    \&\& {\mult} \\
                    \&\&
                    \arrow[no head, from=3-3, to=4-3]
                    \arrow[from=2-4, to=3-3]
                    \arrow[from=1-3, to=2-4]
                    \arrow[from=1-5, to=2-4]
                    \arrow[from=1-1, to=3-3]
                \end{tikzcd}
                \hspace{2em}
                % https://q.uiver.app/#q=WzAsNixbMiwyLCIrIl0sWzIsM10sWzIsMCwiMSJdLFs0LDAsIjEiXSxbMCwwLCIyIl0sWzEsMSwiKyJdLFswLDEsIiIsMCx7InN0eWxlIjp7ImhlYWQiOnsibmFtZSI6Im5vbmUifX19XSxbMywwXSxbNCw1XSxbMiw1XSxbNSwwXV0=
                \begin{tikzcd}[ampersand replacement=\&,cramped]
                    x \&\& y \&\& z \\
                    \& {\mult} \\
                    \&\& {\mult} \\
                    \&\& {}
                    \arrow[no head, from=3-3, to=4-3]
                    \arrow[from=1-5, to=3-3]
                    \arrow[from=1-1, to=2-2]
                    \arrow[from=1-3, to=2-2]
                    \arrow[from=2-2, to=3-3]
                \end{tikzcd}
            }
        \end{center}
    \end{example}
\end{toappendix}
%
Finally, we say how a given $\sig$-structure $\str{X}$
\emph{satisfies} the system of equations $T_{(\sig,\eqsig)}$.
%
We need to assign a value to each free variable in the equation
in the carrier set, using a valuation function $\rho : V \to X$.
%
Given such an assignment, we evaluate the left and right trees of the equation,
by extending $\rho\,$ with~\cref{def:universal-extension},
that is by construction, a homomorphism from $\str{F}(V)$ to $\str{X}$.
%
To satisfy an equation, these two evaluations should agree.
%
\begin{definition}[\alink{definition}{Satisfaction $\str{X} \entails T$}]
    A $\sig$-structure $\str{X}$ satisfies the system of equations $T_{(\sig,\eqsig)}$ if for every set $V$,
    and every assignment $\rho : V \to X$, $\ext{\rho}$ is a (co)fork:
    % https://q.uiver.app/#q=WzAsMyxbMiwwLCJcXHN0cntGfShWKSJdLFs0LDAsIlxcc3Rye1h9Il0sWzAsMCwiXFxFcVNpZyhWKSJdLFswLDEsIlxcZXh0e1xccmhvfSJdLFsyLDAsIlxcZXFsZWZ0X3tWfSIsMCx7Im9mZnNldCI6LTN9XSxbMiwwLCJcXGVxcmlnaHRfe1Z9IiwyLHsib2Zmc2V0IjozfV1d
    \[\begin{tikzcd}[ampersand replacement=\&,cramped]
            {\EqSig(V)} \&\& {\str{F}(V)} \&\& {\str{X}}
            \arrow["{\ext{\rho}}", from=1-3, to=1-5]
            \arrow["{\eqleft_{V}}", shift left=3, from=1-1, to=1-3]
            \arrow["{\eqright_{V}}"', shift right=3, from=1-1, to=1-3]
        \end{tikzcd}\]
\end{definition}
%
Given a signature $(\sig,\eqsig)$ with a system of equations $T_{(\sig,\eqsig)}$,
the $\sig$-algebras satisfying $T_{(\sig,\eqsig)}$, or varieties, form a full subcategory of $\sigAlg$.
%
While free algebras exist for any signature $\sig$ as in~\cref{def:free-algebra-construction},
constructions of free varieties for arbitrary systems of equations require non-constructive
principles~\cite[\S~7, pg.142]{blassWordsFreeAlgebras1983},
in particular, the arity sets need to be projective -- we do not pursue this matter further.
%
% The non-constructive principles can be avoided if we limit ourselves to specific constructions
% where everything is finitary. Of course, HoTT/UF offers an alternative by allowing higher generators
% for equations using HITs~\cite{univalentfoundationsprogramHomotopyTypeTheory2013}.
%
We have a rudimentary framework for universal algebra and equational logic,
which gives us enough tools to develop the next sections.
